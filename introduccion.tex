\secnumberlesssection{INTRODUCCIÓN}

Dentro de la física computacional, uno de los mayores desafíos es el poder desarrollar métodos computacionales que puedan simular los fenómenos físicos de la manera más precisa posible. Esto suele derivar en un problema de \textit{trade-off}, donde uno tiene que sacrificar grados de libertad por viabilidad computacional.

En las últimas décadas, se ha experimentado un desarrollo de múltiples métodos que permiten que este \textit{trade-off} no sacrifique la calidad de los resultados de las simulaciones y el tipo de sistemas que se pueden resolver. En los últimos años, se ha visto un aumento en el uso de redes tensoriales \cite{Hauschild_2018} \cite{SCHOLLWOCK201196} para el estudio de sistemas físicos \cite{tensor1} \cite{tensor2} y de \textit{machine learning} (\textit{deep learning}) para la estimación de características \cite{MachineLearning1J}\cite{MachineLearning2}.

Frente a este panorama es que otro método ha surgido como solución, el cual trae un nuevo paradigma, para llevar a cabo simulaciones, que es la computación cuántica. Durante los últimos años, ha adquirido gran relevancia debido a los progresos de grandes compañías como IBM y Google, quienes han desarrollado procesadores de 50 a 200 \textit{qubits} (\textit{sycamore} e \textit{IBM eagle}) \cite{Stanisic2022} \cite{Powers2023}(Esto ha fomentado diversas nuevas investigaciones \cite{Stanisic2022} \cite{Powers2023} de diverso carácter que muestran el potencial de la computación cuántica.

En el presente trabajo se propone y estudia un flujo de trabajo/cálculo para la implementación y utilización de algoritmos cuánticos para el estudio de diversos modelos de la física del estado sólido, la materia condensada y la química. Para ello, se utilizan librerías especializadas en el trabajo de circuitos cuánticos, además de una revisión del estado actual del arte para trabajar con un conjunto de hiperparámetros adecuados para cada modelo. El algoritmo central en el desarrollo de la memoria es el método VQE (junto a un par de variaciones), el cual es puesto a prueba en diferentes modelos cuyos estados tienen diferentes niveles de correlaciones (haciendo del cálculo más o menos complejo), con el objetivo de mostrar su viabilidad y potencial.

El escrito se divide en cinco capítulos con un anexo. Los primeros dos capítulos consisten en la presentación del problema y conceptos clave para entender el desarrollo realizado (marco conceptual). En el tercer capítulo se presenta la solución propuesta con los límites y supuestos considerados en el desarrollo. El cuarto capítulo se divide en cuatro subcapítulos, en cada uno se analizan características distintas (las cuales requieren metodologías distintas que son presentadas antes de los resultados). Finalmente, en el capítulo cinco se presentan las conclusiones en cuatro subcapítulos, el primero, son unas palabras acerca de los resultados obtenidos, después, un análisis sobre los objetivos y los últimos son el trabajo futuro y las palabras finales del autor. En el anexo se presenta el enlace al repositorio del proyecto junto a los datos obtenidos.
